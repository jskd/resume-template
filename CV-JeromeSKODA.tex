%%%%%%%%%%%%%%%%%%%%%%%%%%%%%%%%%%%%%%%%%
% Friggeri Resume Custom
% XeLaTeX Template
% Version 1.3 (11/07/2018)
% Jérôme Skoda <contact@jeromeskoda.fr>
%
% This template has been downloaded from:
% https://github.com/jskd/resume-template
%
% Original author:
% Adrien Friggeri (adrien@friggeri.net)
% https://github.com/afriggeri/CV
%
% Modifications by Matthew Davis (matthew@mdavis.xyz)
% Version 1.2 (3/5/15)
% https://github.com/mlda065/friggeri-letter
%
% License:
% CC BY-NC-SA 3.0 (http://creativecommons.org/licenses/by-nc-sa/3.0/)
%
% Important notes:
% This template needs to be compiled with XeLaTeX
%
%%%%%%%%%%%%%%%%%%%%%%%%%%%%%%%%%%%%%%%%%
\documentclass[]{friggeri-cv-custom} % Add 'print' as an option into the square bracket to remove colors from this template for printing
\begin{document}
\header{Jérôme~}{S\textsc{koda}}{Passionné d'informatique et de nanosatellite} % Your name and current job title/field
%----------------------------------------------------------------------------------------
%	SIDEBAR SECTION
%----------------------------------------------------------------------------------------
\begin{aside}
  XX ans
  \section{Contact}
  Adresse
  Code postal Ville
  ~~
  {\Large\faMobilePhone}~~XX XX XX XX
  \faEnvelopeO~~\href{mailto:contact@jeromeskoda.fr}{contact@jeromeskoda.fr}
  \faGlobe~~\href{https://www.jeromeskoda.fr}{jeromeskoda.fr}
  \faGithub~~\href{https://www.github.com/jskd}{github.com/jskd}
  \faLinkedin~~\href{https://www.linkedin.com/in/jeromeskoda}{linkedin.com/in/jeromeskoda}
  \section{Langages}
  \exphig{C}~\textbullet~\exphig{C++}~\textbullet~\explow{Assembleur}
  \expmed{VHDL}~\textbullet~\explow{Verilog}
  \expmed{Java}~\textbullet~\explow{Scala}~\textbullet~\expmed{Kotlin}~\textbullet~\expmed{\Csharp}
  \exphig{Python}~\textbullet~\expmed{Javascript}~\textbullet~\exphig{Bash}
  \expmed{PHP}~\textbullet~\exphig{HTML}~\textbullet~\exphig{CSS}
  \expmed{SQL}~\textbullet~\explow{OCaml}~\textbullet~\expmed{\LaTeX}
  \section{Programmation}
  \exphig{Orientée~objet}~\textbullet~\expmed{Concurrente}
  \explow{Fonctionnelle}~\textbullet~\explow{Logique}
  \section{Environnement}
  \exphig{Système embarqué (ARM)}
  \expmed{FPGA}~\textbullet~\expmed{NIOS}~\textbullet~\expmed{FreeRTOS}
  \exphig{GNU/Linux}~\textbullet~\expmed{Windows}
  \expmed{Android}~\textbullet~\expmed{Windows Phone}
  \section{Versionnage}
  \exphig{Git}~\textbullet~\expmed{SVN}~\textbullet~\explow{Mercurial}
  \section{Editeurs}
  \exphig{Eclipse}~\textbullet~\expmed{Visual Studio}
  \expmed{IntelliJ}~\textbullet~\exphig{Atom}~\textbullet~\exphig{Vim}
  \expmed{Quartus II}~\textbullet~\expmed{Altium Designer}
  \section{Outils}
  \exphig{JIRA}~\textbullet~\expmed{Confluence}
  \exphig{Jenkins}~\textbullet~\expmed{Travis CI}~\textbullet~\explow{SonarQube}
  \expmed{GDB}~\textbullet~\expmed{Valgrind}~\textbullet~\explow{Strace}
  \expmed{GCC}~\textbullet~\explow{Clang}~\textbullet~\expmed{Make}
  \section{SGBD}
  \exphig{MySQL}~\textbullet~ \expmed{PostgreSQL}
  \explow{MongoDB}
  \section{Web}
  \expmed{Jquery}~\textbullet~\explow{Gulp}~\textbullet~\expmed{Node.js}
  \expmed{Symfony}~\textbullet~\explow{WordPress}~\textbullet~\explow{Play}
  \section{Centres d'intérêt}
  Natation~\textbullet~Informatique
  Jeu de société
\end{aside}
%----------------------------------------------------------------------------------------
%	EDUCATION SECTION
%----------------------------------------------------------------------------------------
\vspace{1.75\parskip}
\section{Formation}
\begin{entrylist}
%------------------------------------------------
\entry
{Master Informatique}
{2018}
{Université Paris VII - Paris Diderot}
{Mention{:} Bien}
{
	Machine Learning~\textbullet~
	Informatique embarquée~\textbullet~
	Programmation orientée objet~\textbullet~
	Interface graphique~\textbullet~
	Programmation système~\textbullet~
	Ingénierie des Protocoles Réseau~\textbullet~
	Bases de données~\textbullet~
	Modélisation et spécification~\textbullet~
	Programmation mobile
}
%------------------------------------------------
\entry
{Licence Informatique}
{2016}
{Université Paris VII - Paris Diderot}
{Mention{:} Assez bien}
{
  Sécurité informatique~\textbullet~
  Langages de script~\textbullet~
  Programmation Web~\textbullet~
  Programmation fonctionnelle~\textbullet~
  Logique~\textbullet~
  Programmation réseau~\textbullet~
  Algorithmique
}
%------------------------------------------------
\entry
{DUT Génie Electrique Informatique Industrielle}
{2015}
{Université Paris XI - IUT de Cachan}
{}
{
	Système numérique~\textbullet~
  Génie logiciel~\textbullet~
	Electronique pour le traitement et la transmission de l'information~\textbullet~
	Robotique~\textbullet~
	Traitement numérique du signal
}
\end{entrylist}%
%----------------------------------------------------------------------------------------
% EXPÉRIENCE PROFESSIONELLE
%----------------------------------------------------------------------------------------
\section{Expérience professionelle}
\begin{entrylist}
%------------------------------------------------
\entry
{Ariane6 Préparation des missions - Développement Python}
{Avril-Septembre 2018}
{ArianeGroup}
{}
{
  Réalisation d'un système de validation automatique basé sur JIRA et Jenkins.
  Conçu pour être modulaire et adaptable, il permet de tester des sources
  en Python, CShell et Fortran.
  Conception complète en cycle en V de l'annalyse du besoin
  jusqu'à la validation.
}
%------------------------------------------------
\entry
{IGOSat - Développement C sur ARM}
{Juin-Juillet 2017}
{Université Paris VII - Paris Diderot - Centre Spatial Étudiant}
{Collaboration bénévole}
{
  Retour d'expérience de mes autres contributions sur projets de Cubesat.
  Mise en place d'une chaîne de compilation complète basé sur Make et GCC pour
  obtenir une maîtrise de complète du processus
  de compilation. Soutiens du développement logiciel de vol d'IGOSat.%
}
%------------------------------------------------
\entry
{EyeSat - Développement VHDL/C++ sur FPGA/NIOS}
{Avril-Juillet 2016}
{IUT de Cachan - Innov'Lab}
{Lancement prévu entre octobre 2018 et mars 2019}
{
  Réalisation d'une caméra spatiale: conception du PCB supportant le capteur
  d'image avec Altium Designer; programmation d'un FPGA (Cyclone II d'Altera)
  en co-design VHDL/C++ dialoguant avec le capteur et réalisation
  d'une interface de récupération des informations sur ordinateur.
}
%------------------------------------------------
\entry
{QB50 (XcubeSat/SpaceCube) - Développement C++ sur ARM}
{Avril-Juillet 2015}
{Ecole polytechnique - Centre Spatial Etudiant}
{Mis en orbite en mai 2017}
{
  Implémentation logicielle du protocole AX.25 permettant de transmettre les
  télémesures d'un satellite vers la station sol et conception d'une interface
  de test embarqués. Développement en C++ sur l'ordinateur de bord composé d'un
  ARM Cortex M4 avec FreeRTOS.
}
\end{entrylist}%
%----------------------------------------------------------------------------------------
% EXPERIENCE SECTION
%----------------------------------------------------------------------------------------
\section{Projet personnel}
\begin{entrylist}
%------------------------------------------------
\entry
{Cards Save Humanity}
{Juin 2018 à aujourd'hui}
{Application native Android développée en Kotlin}
{Finaliste du hackathon SOGETI GREEN X GAMES}
{
  Jeu sur le thème du développement durable où l'on doit prendre des décisions
  tout en maintenant un équilibre entre les différents axes du développement
  durable: l'environnement, l'industrie et le social. Projet en cours de
  développement avec une sortie prévue sur le Play~Store.%
}
%------------------------------------------------
\entry
{Dictionnaire de la langue des signes française}
{Mars - Avril 2015}
{Application multiplate-forme en \Csharp /XAML}
{Plus de 30 000 téléchargements et 150 avis}
{
  Application permettant d'apprendre la langue des signes française (LSF).
  Construite comme un dictionnaire, cette application possédant plus de 800
  signes interprétés par des acteurs.
  Conçue en ``Universal Windows Apps'', elle est compatible sur tous les appareils
  Windows à partir de la version 8.1 (PC/Windows Phone). Disponible
  gratuitement sur le Windows Store.%
}%
\end{entrylist}%
\about{Consulter mon portfolio sur: \href{https://www.jeromeskoda.fr}{www.jeromeskoda.fr}}%
%------------------------------------------------
\end{document}
